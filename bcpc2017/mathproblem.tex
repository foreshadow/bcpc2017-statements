\gdef\thisproblemauthor{Griffin}
\begin{problem}{数学难题}
{标准输入}{标准输出}
{1 s}{64 MB}{}

Griffin:``啥?校赛要出题。可是我好菜啊!''\\
骚年,这还不简单,随便出道数学题就好啊?\\
于是Griffin就查看了一下某著名数字序列百科全书网站,他发现了一个interesting的问题,
叫``The Minimal Superpermutation Problem''问题。
这个问题可以表述为:\\
在字典 $\{1,2,\ldots ,n\}$上,对于一个字符串,要求其 $1$ 到 $n$ 的 $n!$ 个排列都是其子串,求这样的字符串的最短长度。\\
同学A:"喵喵喵?就不能说得通俗一点吗?"\\
那好吧,举个例子,假设有一个 $n$ 集的连续剧,对应 $n$ 张DVD,每张DVD不可区分,也不能通过剧情内容区分DVD集数,
问至少要看多少张DVD才能\textbf{确保一定}以正确顺序连续观看一遍该连续剧?\\
For example,假设现在 $n=2$ ,我们将两张DVD命名为1,2,我们以121的顺序观看,
这样无论1和2哪个是真正的第一集都能保证我们以最小的次数3连续观看完正确顺序的剧情。

\InputFile

第一行有一个整数 $t\ (1 \le t \le 100)$,表示有 $t$ 组数据。\\
对于每组数据:\\
第一行为一个正整数 $n\ (1\le n\le 1000)。$

\OutputFile

对于每组数据,输出一个整数,表示最短的长度。结果可能很大,请对 $10^9+7$ 取模。

\Example

\begin{example}
\exmp{
2
1
2
}{%
1
3
}%
\end{example}

\end{problem}
