\gdef\thisproblemauthor{wchhlbt}
\begin{problem}{老顽童锻炼计划}
{标准输入}{标准输出}
{1 s}{64 MB}{}

老顽童是一个热爱运动的人,每天都有小顽童来和他一起锻炼。为了让锻炼更有成效,
他想要制定一份锻炼计划,让第 $n$ 天的训练量要等于第 $n$ 个正回文数(第1个正回文数是1)。\\
老顽童已经打印好了计划表,但是他的计划表中有一些日子被遗漏了,他又不想重新计算,你可以帮他解决这个问题吗?\\
如果一个数的各位数字反向排列后仍然等于其本身,我们把这样的数称为回文数(palindrome number)。
例如:12321是回文数,1232不是回文数。

\InputFile

第一行有一个整数 $t\ (1 \le t \le 100000)$,表示有 $t$ 组数据。\\
对于每组数据:\\
接下来 $t$ 行,每一行有一个整数 $n\ (1\le n\le 400000)$,表示第 $n$ 天。

\OutputFile

对于每组数据,输出一个整数,表示对应的训练量。\\

\Example

\begin{example}
\exmp{
5
1
2
3
12
23
}{%
1
2
3
33
141
}%
\end{example}

\end{problem}
