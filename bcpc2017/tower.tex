\gdef\thisproblemauthor{Voleking}
\begin{problem}{Tower of Hanoi}
{标准输入}{标准输出}
{1 s}{64 MB}{}

变种汉诺塔问题和普通汉诺塔问题略有不同,规则描述如下:
\begin{enumerate}
  \item 有三根柱子,在最左侧柱子上放置着若干圆盘。与传统汉诺塔不同的是,其中存在部分大小相同的圆盘。
  \item 要求包括初始状态在内,每个圆盘上方放置的圆盘不得大于该圆盘,即圆盘上方只能放置小于自己或和自己相同大小的圆盘。
  \item 每次移动只能将某柱子最顶部的一个圆盘移动到另一柱子的最顶部。
  \item 需要注意的是,大小相同的圆盘具有的其他特征是不一样的,例如不同颜色。
\end{enumerate}
最后需要保证2号柱子上的圆盘排列顺序,和开始时的0号柱子上的顺序完全相同。\\
求将初态0号柱子上的所有圆盘全部移到2号柱子上最优策略的步数 $l$对 $m$ 取模后的值。\\

\InputFile

第一行有一个整数 $t\ (1 \le t \le 100)$,表示有 $t$ 组数据。\\
对于每组数据:\\
第一行包括2个数字 $n,m\ (1\le n\le 15000,1\le m\le 1000000)$,其中 $n$ 代表圆盘种类的个数;\\
第二行包括 $n$ 个数字 $a_1,…,a_n\ (1\le a_i\le 99)$,其中 $a_i$ 代表大小为 $i$ 的圆盘个数。

\OutputFile

对于每组数据,输出一行,若最优策略的步数为 $l$,则输出 $l\ \text{mod}\ m$。

\Examples

\begin{example}
\exmp{
2
2 1000
1 2
3 1000
1 2 3
}{%
7
21
}%
\end{example}

\end{problem}
